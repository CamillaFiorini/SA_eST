\documentclass[10pt]{article}
\usepackage[usenames]{color} %used for font color
\usepackage{amssymb} %maths
\usepackage{amsmath,graphicx} %maths
\usepackage[utf8]{inputenc} %useful to type directly diacritic characters
\usepackage{showkeys} %maths
\addtolength{\hoffset}{-1.5cm}
\addtolength{\textwidth}{3.cm}
\addtolength{\voffset}{-1.cm}
\addtolength{\textheight}{2.0cm}


\title{In-cell shock resolution: tracking vs sub-cell reconstruction. Application to sensitivity analysis}
\author{MC-CF-MR}
\begin{document}


\maketitle
\tableofcontents


\section{General strategy}

We consider the solution of a hyperbolic system of conservation laws: 
 \begin{equation}\label{eq:1}
\partial_tU+ \partial_x F(U) =0
   \end{equation}
by a classical FV approach with a first order explicit update written as
 \begin{equation}\label{eq:2}
\Delta x\dfrac{\bar U^{n+1}_i-\bar U^{n}_i}{\Delta t}+ \hat F_{i+1/2}  -  \hat F_{ii1/2} =0
   \end{equation}
   where   $\bar U_i^n$ denotes  the discrete approximation of the average of $U$ over cell $i:=[x_{i-1/2},\,x_{i+1/2}]$ at time $t^n$,
   and where $\Delta x =x_{i+1/2}-x_{i-1/2}$, and $\Delta t=t^{n+1}-t^n$.
As usual $\hat F_{i+1/2} $ denotes a numerical flux function of the left and right states $(U_{i+1/2}^-,\;U_{i+1/2}^+)$.

For later use we also recast  \eqref{eq:2} 
 

\section{Tracking method}
\section{In-cell recontruction}
 
\section{Sensitivity equation: generalities}

\section{Numerical implementation}

\section{Results}


 

 \end{document}